\documentclass[10pt,journal,compsoc]{IEEEtran}


\newcommand\MYhyperrefoptions{bookmarks=true,bookmarksnumbered=true,
	pdfpagemode={UseOutlines},plainpages=false,pdfpagelabels=true,
	colorlinks=true,linkcolor={black},citecolor={black},urlcolor={black},
	pdftitle={Review of the health benefits Vs the Side effects of using Virtual Reality Technology},%
	pdfsubject={Typesetting},%
	pdfauthor={Michael Kidd},%
	pdfkeywords={template}}%

\hyphenation{op-tical net-works semi-conduc-tor}

\begin{document}
	
	\title{Review of using virtual reality technology in the medical industry}
	
	\author{Michael~Kidd~- G00236920%
		\IEEEcompsocitemizethanks{\IEEEcompsocthanksitem Michael Kidd is a student with Galway 
			E-mail: G00236920@gmit.ie
		\IEEEcompsocthanksitem Lecturer John French - John.French@gmit.ie.}%
	
		\thanks{Manuscript Sent as an Assignment 14th December 2018.}}
	
	% The paper headers
	\markboth{Literature review for Research Methods in Software Development}%
	{Shell \MakeLowercase{\textit{et al.}}: Review of using Virtual Reality Technology in the medical industry}
	
	\IEEEtitleabstractindextext{%
		\begin{abstract}
			This document is a Literature review on research regarding the health benefits and side effects from using Virtual reality technology in 3D Environments for training medical professionals and in helping people with medical conditions. The use of virtual reality technology has become more widespread over the last few years. Now the range of virtual reality technology has increased from the use of mobile devices mounted in a plastic or cardboard mounted headset, to specifically designed units that are used to play video games, train personnel in different fields and assist with different medical conditions, both physical and mental. The aim is to exam the benefits and side effects presented by continued use of these devices for training and treatment. 
		\end{abstract}
		
		\begin{IEEEkeywords}
				Keyword here, index terms
		\end{IEEEkeywords}}
	
	
	\maketitle
	
	\section{Introduction}
	

	\IEEEPARstart{T}{his} document is a review of the health benefits and side effects of prolonged or continued use of virtual reality head mounted display units for training medical personnel and treating patients with medical conditions. With the introduction of HMD units to the mass market, through companies such as Facebook with the oculus line of units, HTC with the Vive, Samsung with the Odyssey and many other groups and start-ups that have designed their own devices and equipment, the amount of users is only set to increase at the moment. With an increase of users it is only natural to wonder and question if using these devices is beneficial or hazardous to the people who use them and whether or not they help to train people more effectively and if they assist with the treatment of physical and mental conditions. \newline \newline
	Head mounted display units and room scale environments allow many users to experience many different things and locations that otherwise would be impossible. This provides a unique opportunity to introduce an environment that would be difficult or expensive to simulate in the real world. Some countries use virtual reality to train medical personnel in war like situations, with the intention of retaining knowledge and training that can only be attained by repeatedly performing a task. This same idea can be applied in many fields,  medical students could be trained on the inner workings of the human body without the need for human demonstrations. These simulations can be run repeatedly, they can be paused and altered and changed to fit the target audience. If an picture speaks a thousand words, then what can a 3D environment that can be paused, rerun, reshaped and seen from many points of view be worth as a teaching tool.
	\newline
	
	
	
	\section{Training medical professionals} \newline
	
	Surgery within military and civilian situations differ greatly, a civilian surgeon may be highly skilled in a hospital were surgery is expected and prepared for, the surgeons are also assisted by other medical professionals. In a war zone a surgeon with the same training would also require other skills to meet the tasks at hand. In a civilian environment the surgeon would be in an operating theatre were the conditions are sterile and free from interruptions, were as a military surgeon would be in any number of locations and uncontrollable situations. These are unpredictable and cannot be practised without great expense. \newline \newline
	Training in virtual simulations is now widespread in the U.S. military (Siu Best, Kim, Oleynikov, Ritter, 2016, p. 215) but is limited to simpler surgical procedures such as laparoscopic surgeons (surgery that requires a telescopic camera incisions) and sutures, this could however be expanded to tasks that require a higher skill set and situations that occur less frequently but were success is critical when they do occur. Along with the training of the overall subject or task virtual simulations allow for better training of cognitive skills for the more basic tasks involved that often are overlooked. \newline
	An advantage to using virtual simulations is that it allows for intervention before bad habits have developed. 
	
	
	\section {The Skill Retention Theory}
	
	
	
	\subsection{Subsection Heading Here}
	
	\subsubsection{Subsubsection Heading Here}
	
	
	\section{Conclusion}
	
	
	
	\appendices
	\section{Proof of the First Zonklar Equation}

	\section{}
	
	
	\ifCLASSOPTIONcompsoc

	\section*{Acknowledgments}
	\else

	\section*{Acknowledgment}
	\fi
	

	\begin{IEEEbiography}{Michael Kidd}
		
	\end{IEEEbiography}

	\begin{thebibliography}{1}
		
		\bibitem{IEEEhowto:kopka}
		H.~Kopka and P.~W. Daly, \emph{A Guide to {\LaTeX}}, 3rd~ed.\hskip 1em plus
		0.5em minus 0.4em\relax Harlow, England: Addison-Wesley, 1999.
		
	\end{thebibliography}
	
\end{document}


