\documentclass[10pt,journal,compsoc]{IEEEtran}


\newcommand\MYhyperrefoptions{bookmarks=true,bookmarksnumbered=true,
	pdfpagemode={UseOutlines},plainpages=false,pdfpagelabels=true,
	colorlinks=true,linkcolor={black},citecolor={black},urlcolor={black},
	pdftitle={Review of the health benefits Vs the Side effects of using Virtual Reality Technology},%
	pdfsubject={Typesetting},%
	pdfauthor={Michael Kidd},%
	pdfkeywords={template}}%

\hyphenation{op-tical net-works semi-conduc-tor}

\begin{document}
	
	\title{Review of using virtual reality technology in the medical industry}
	
	\author{Michael~Kidd~- G00236920%
		\IEEEcompsocitemizethanks{\IEEEcompsocthanksitem Michael Kidd is a student with Galway 
			E-mail: G00236920@gmit.ie
		\IEEEcompsocthanksitem Lecturer John French - John.French@gmit.ie.}%
	
		\thanks{Manuscript Sent as an Assignment 14th December 2018.}}
	
	% The paper headers
	\markboth{Literature review for Research Methods in Software Development}%
	{Shell \MakeLowercase{\textit{et al.}}: Review of using Virtual Reality Technology in the medical industry}
	
	\IEEEtitleabstractindextext{%
		\begin{abstract}
			This document is a Literature review on research regarding the benefits of using Virtual reality technology in 3D Environments for training medical professionals and in helping people with medical conditions. The use of virtual reality technology has become more widespread over the last few years. Now the range of virtual reality technology has increased from the use of mobile devices mounted in a plastic or cardboard mounted headset, to specifically designed units that are used to play video games, train personnel in different fields and assist with different medical conditions, both physical and mental. The aim is to examine the benefits and side effects presented by continued use of these devices for training and treatment. 
		\end{abstract}
		
		\begin{IEEEkeywords}
				Keyword here, index terms
		\end{IEEEkeywords}}
	 
	\maketitle
	
	\section{Introduction}
	\IEEEPARstart{T}{he} premise of a literature review is to pose a specific question, the question that should be addressed here is, Is virtual reality technology viable as a teaching or treatment tool in the medical community. This document is a review of the benefits of using virtual reality head mounted display units for training medical personnel and treating patients with mental medical conditions. With the introduction of HMD units to the mass market, through companies such as Facebook with the oculus line of units, HTC with the Vive, Samsung with the Odyssey and many other groups and start-ups that have designed their own devices and equipment, the amount of users is only set to increase at the moment. With an increase of users it is only natural to wonder and question if using these devices is beneficial or hazardous to the people who use them and whether or not they help to train people more effectively and if they assist with the treatment of physical and mental conditions. \newline
	
	Head mounted display units and room scale environments allow many users to experience many different things and locations that otherwise would be impossible. This provides a unique opportunity to introduce an environment that would be difficult or expensive to simulate in the real world. Some countries use virtual reality to train medical personnel in war like situations, with the intention of retaining knowledge and training that can only be attained by repeatedly performing a task. This same idea can be applied in many fields,  medical students could be trained on the inner workings of the human body without the need for human demonstrations. These simulations can be run repeatedly, they can be paused and altered and changed to fit the target audience. If an picture speaks a thousand words, then what can a 3D environment that can be paused, rerun, reshaped and seen from many points of view be worth as a teaching tool.
	\newline
		
	\section{Training medical professionals}	
	Surgery within military and civilian situations differ greatly, a civilian surgeon may be highly skilled in a hospital were surgery is expected and prepared for, the surgeons are also assisted by other medical professionals. In a war zone a surgeon with the same training would also require other skills to meet the tasks at hand. In a civilian environment the surgeon would be in an operating theatre were the conditions are sterile and free from interruptions, were as a military surgeon would be in any number of locations and uncontrollable situations. These are unpredictable and cannot be practised without great expense. \newline
	
	Training in virtual simulations is now widespread in the U.S. military (Siu Best, Kim, Oleynikov, Ritter, 2016, p. 215) but is limited to simpler surgical procedures such as laparoscopic surgeons (surgery that requires the use of a telescopic camera) and sutures, this could however be expanded to tasks that require a higher skill set and situations that occur less frequently but where success is critical when they do occur. Along with the training of the overall subject or task virtual simulations allow for better training of cognitive skills for the more basic tasks involved that often are overlooked. \newline
	An advantage to using virtual simulations is that it allows for intervention before bad habits have developed. 
	
	\subsection{The skill retention theory}
	(Siu Best, Kim, Oleynikov, Ritter, 2016, p. 216) mentions the skills retention theory, were there are 3 stages of learning any task. \newline
	
	Stage  1: Declarative learning, this stage involves the basic learning of a skill and can degrade with lack of use to a point were the learner may be unable to perform the task. In this learning phase steps can be missed. \newline 
	
	Stage 2: Associative  learning, the learner has learned the fundamentals of the topic or task but this mixed knowledge. In stages 1 and 2 training should be introduced to keep the knowledge active.\newline
	
	Stage 3: Procedural learning, is where we have the statement "Practice makes perfect". When the learner has performed a task they truly understand and has practised to a point that the they no longer rely on declarative memory but procedural memory. \newline
	
	When it comes to training tasks that either would be expensive to replicate or just impossible for many possible reasons, Virtual reality training simulations could become invaluable. Allowing a learner to repeat a task at their own discretion, to build the procedural memory needed to master a task, to build the muscle memory that would otherwise take time to develop in a real environment and to be free of the consequences of failure during training, on the other hand this can also be a negative as the consequences when a real world situation presents itself can be far greater and must also be understood. Similar to a child playing a racing car video game in an arcade simulation is in no way prepared to drive a car, it does however make them familiar with the controls and demonstrate a basic understanding of the task. \newline
	
	(Siu Best, Kim, Oleynikov, Ritter, 2016, p. 216), This publication was published in the Military medicine journal in may of 2016 volume 181. It also contained insights on an experiment they performed using 5 novices and 4 medical trainees. Each were given basic tasks to perform over a number of weeks. The data from the skill retention seems to be valid. The idea behind the decay of knowledge involved in performing basic tasks is very interesting and shows that using VR technology to assist with teach those tasks shows an improvement compared to traditional methods. The limitation of the experiment is made with a very small sample size of 9 people, so the results may not be completely relevant in a larger sample size. One area of research that would be appropriate to follow this document would be to test the method using a larger sample size but also run an experiment with tasks that 
	
	\section{Virtual Reality studies to improve mental health}
	 On 16 November 1973 Skylab 4 was launched and concluded on 8 February 1974, it was the longest space flight at that time lasting 81 days. However unlike many manned space missions before it, Skylab 4 was the first "Space mutiny". Gerald Carr, William Pogue and Edward Gibson the astronauts tasked with the mission refused to perform their duties for 24 hours due to the high stress environment and heavy work load they endured. (Salamon et al., 2018) uses this example as a reference when suggesting applications that would assist in preparing astronauts for manned space flight. This incident brought attention to the need to support the mental health as well as the physical health of the astronauts. When it has comes to space flight in the past, Space exploration organisations have relied on personnel from military backgrounds that have scientific training to train as astronauts as these individuals fundamentally are disciplined and follow a code of conduct that demands they follow the chain of command, they are also often able to withstand situations that place them under great amounts of stress. Virtual reality simulations and interactions could prove to be an invaluable tool for manned space flights as it would allow for a detraction from the reality of being trapped within a confined space often without human contact of any kind. \newline 
	 
	 This article was published May 1, 2018, in Acta Astronautica. The document at this moment has no citations but was an interesting read. It does have an interesting approach to the issues that face astronauts in longer manned space missions, However presents many gaps in the idea behind how the system would work or the options that the astronauts would require over a longer period of time and offers no implementation methods.\newline
	 
	 (Maples-Keller et al., 2017) suggests using Virtual reality for phobia based treatments that would for example allow a person who is afraid of flying to face that fear in a controlled environment. The patient may not be ready to face turbulence or may even be unable to step onto a plane without placing other passengers at risk. A phobia of heights could also be addressed using VR technology with no possibility of a fall. Of course these technologies would still require conventional treatments and for the phobias to be address would require an actual experience of the phobia condition but this technology would be a stepping stone that could help with many difficulties that initially would prevent treatment. (Salamon et al., 2018) also suggests that VR could be used to threat such mental disorders as Social anxiety disorder were the patient experiences anxieties when in conversations, meeting new people or public speaking, with VR these experiences could be  gradually and could stopped at a push of a button, this would prevent further trauma to the patient. It also references a article (Reger GM, Koenen-Woods P, Zetocha K, et al. 2016), A controlled experiment that was performed with active soldiers who suffered from PTSD and 9/11 survivors who suffered from PTSD, the results showed an improvement from VR environments over other treatments including placebo treatments. Panic disorder and agoraphobia are also covered and show an interesting point regarding agoraphobia, When someone has difficulties with open spaces and crowds. Virtual reality is an obvious path that can be taken in order to threat agoraphobia as it can allow sufferers to deal with their phobia in smaller steps while also being in a safe environment and the equipment can be removed. \newline
	 
	 This article comes from the Department of Psychiatry and Behavioural Sciences, Emory University School of Medicine. It offers a breakdown and explanation of each of the disorders described, it also shows a table of an experiment that shows the conditions to qualify to be part of the test groups. It also contains a table depicting the studies and conditions of the individual disorders tested and the conditions of the treatment used and age groups tested. \newline

	\begin{quotation}
		 Most research carried out before 2012 focused on anxiety dis-orders (Opris et al., 2012), eating disorders (Ferrer-Garcia et al., 2013), phobias (Botella et al., 2014) and post traumatic stress disorder (DiMauro, 2014). Findings showed the effectiveness of VR compared to treatment as usual, but only small effect sizes when VR was compared to conventional cognitive behaviour therapy (Eichenberg and Wolters, 2012).\newline 
		 
		 (Maples-Keller et al., 2017)
	\end{quotation}

\section{Cyber Sickness}
Cyber sickness is an effect brought on by motions the subject is unprepared for and has the same symptoms as motion sickness, Dizziness, nausea, cold sweats and eye strain (Nalivaiko et al., 2015) examines the effects of cyber sickness brought on by the use of virtual reality headsets. The article references Reason and Brands "Sensory Conflict", which determines that motion sickness is caused when the human eye sends motion signals to the brain, when these signals do not match the subjects movement or the movement they expect. When virtual reality is used, the visuals experienced by the subject are completely different from those experienced by the subjects body, fo instance the visual could depict a view of ffree falling however th subject would be firmly planted on the ground. Th disassociation felt by the user is the main reason for the cyber sickness that afflicts the subject. \newline

(Nalivaiko et al., 2015) A experiment was performed with 26 people 18 male, 8 female between the ages of 18-30 Separated into to two groups(9 men, 4 women). During the experiment, the temperature of the subjects finger and forearm were tested while they were placed in a room at 22 degrees Celsius for ten minutes before the experiment began. The subjects then performed 40 trials each, when they saw a cross on a screen they were to click a button on a keyboard. This would calculate their response time and give a more accurate result. The subject were then placed in a roller-coaster simulation on the Oculus DK1 Virtual reality headsets. two simulations were used, one was known as helix, the other known as parrot. Helix is a more realistic simulation of a roller coaster. After 14 minutes the simulation end of if the subject requested the simulation be terminated. During the simulation, every two minutes the rate of nausea was rated by the subjects from 0-10, 0 being none at all and 10 being, ready to vomit. 10 minutes after the simulation, the subjects performed another reaction test.\newline

During the simulation, 67\% of helix subjects terminated the experiment early due to nausea and 17\% from the parrot test group. The conclusion of the experiment was that the temperature of the subjects index finger had a direct correlation with the level of nausea. The article was well documented and the experiment performed was very in-depth and well executed. An improvement to the article cannot be made however a high level overview of the experiment would have made the article easier to read. This is probably due to the target audience of the article and results of the experiment.

\section{Discussion }

As the original question is whether or not Virtual reality would be suited to the task of training medical personnel and treating medical conditions. The side effects of using virtual reality do pose a barrier to training and treatment, however the effects are minimal and could be overcome with longer usage, the effects appear to present within a short time from initial use. Virtual environments may be more suited to seated to tasks.

	\section{Conclusion}
	While researching this topic it was difficult to find some academic sources as the topic itself is very new. The hardware and software capable of this level of immersion in virtual environments was only developed from 2012 and brought to the consumer market in 2016. It would have been preferable to explore the Skylab 4 mutiny in more detail but most academic papers on the subject were not free. As the sources for these topics are only limited to the last few years, the sources do not have large numbers of citations. This may not be an issue as the idea behind researching these topics is to evaluate and discuss the current level of knowledge in the field at this time, even if that level of knowledge is still in its infancy. 

	\section{Acknowledgements}


	\begin{thebibliography}{1}
		
	\bibitem {}
Dyer, E., Swartzlander, B.J., Gugliucci, M.R., 2018. Using virtual reality in medical education to teach empathy. JOURNAL OF THE MEDICAL LIBRARY ASSOCIATION 106, 498–500. https://doi.org/10.5195/jmla.2018.518 \newline
\bibitem {}
Fernández-Álvarez, J., Rozental, A., Carlbring, P., Colombo, D., Riva, G., Anderson, P.L., Baños, R.M., Benbow, A.A., Bouchard, S., Bretón-López, J.M., Cárdenas, G., Difede, J., Emmelkamp, P., García-Palacios, A., Guillén, V., Hoffman, H., Kampann, I., Moldovan, R., Mühlberger, A., North, M., Pauli, P., Peñate Castro, W., Quero, S., Tortella-Feliu, M., Wyka, K., Botella, C., 2018. Deterioration rates in Virtual Reality Therapy: An individual patient data level meta-analysis. Journal of Anxiety Disorders. https://doi.org/10.1016/j.janxdis.2018.06.005\newline
\bibitem {}

Ka-Chun Siu, Best, B.J., Jong Wook Kim, Oleynikov, D., Ritter, F.E., Siu, K.-C., Kim, J.W., 2016. Adaptive Virtual Reality Training to Optimize Military Medical Skills Acquisition and Retention. Military Medicine 181, 214–220. https://doi.org/10.7205/MILMED-D-15-00164\newline
\bibitem {}

Lin, C.J., Widyaningrum, R., 2018. The effect of parallax on eye fixation parameter in projection-based stereoscopic displays. Applied Ergonomics 69, 10–16. https://doi.org/10.1016/j.apergo.2017.12.020\newline
\bibitem {}

Maples-Keller, J.L., Bunnell, B.E., Kim, S.-J., Rothbaum, B.O., 2017. The Use of Virtual Reality Technology in the Treatment of Anxiety and Other Psychiatric Disorders: Harvard Review of Psychiatry 25, 103–113. https://doi.org/10.1097/HRP.0000000000000138\newline
\bibitem {}

Mohr, D.C., Burns, M.N., Schueller, S.M., Clarke, G., Klinkman, M., 2013. Behavioral Intervention Technologies: Evidence review and recommendations for future research in mental health. General Hospital Psychiatry 35, 332–338. https://doi.org/10.1016/j.genhosppsych.2013.03.008\newline
\bibitem {}

Nalivaiko, E., Davis, S.L., Blackmore, K.L., Vakulin, A., Nesbitt, K.V., 2015. Cybersickness provoked by head-mounted display affects cutaneous vascular tone, heart rate and reaction time. Physiology and Behavior 151, 583–590. https://doi.org/10.1016/j.physbeh.2015.08.043\newline
\bibitem {}

Salamon, N., Grimm, J.M., Horack, J.M., Newton, E.K., 2018. Application of virtual reality for crew mental health in extended-duration space missions. Acta Astronautica 146, 117–122. https://doi.org/10.1016/j.actaastro.2018.02.034\newline  
\bibitem {}
Valmaggia, L.R., Latif, L., Kempton, M.J., Rus-Calafell, M., 2016. Virtual reality in the psychological treatment for mental health problems: An systematic review of recent evidence. Psychiatry Research 236, 189–195. https://doi.org/10.1016/j.psychres.2016.01.015\newline
\bibitem {}

Reger GM, Koenen-Woods P, Zetocha K, et al. Randomized controlled trial of prolonged exposure using imaginal exposure vs. virtual reality exposure in active duty soldiers with deployment related post traumatic stress disorder (PTSD). J Consult Clin Psychol. 2016 Advance online publication.\newline 
\bibitem {}

Reason, J.T., 1978. Motion Sickness Adaptation: A Neural Mismatch Model <sup/>. Journal of the Royal Society of Medicine 71, 819–829. https://doi.org/10.1177/014107687807101109\newline


	\end{thebibliography}
	
\end{document}


